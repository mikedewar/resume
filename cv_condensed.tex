\documentclass[line, overlapped]{res}

\usepackage{hyperref}
\hypersetup{colorlinks=true}

\usepackage{libertine}
\renewcommand{\familydefault}{\sfdefault}

\begin{document}

\name{Dr Michael A. Dewar}

\address{
London\\
mikedewar@gmail.com
}

\begin{resume}

I build products based on algorithms, powered by data. I bring advanced
techniques from data science together with mission-driven experts in
order to build next generation products in novel domains.

For a full portfolio, please see \href{https://mikedewar.github.io}{mikedewar.github.io}.

\section{Education}

\emph{The University of Sheffield, UK}: PhD Thesis: \textbf{`A
  Framework for Modelling Dynamic Spatiotemporal Systems'}. Awarded June
  2007. 

\emph{The University of Sheffield, UK}: \textbf{1st Class MEng} in
  Control Systems Engineering. Awarded August 2002.

\section{Employment}

  \emph{January 2018 onwards}: \textbf{Mastercard, Cyber \&
  Intelligence}, Vice President of Data Science.
  \begin{itemize}
	\item Lead a team
	\item Did some M\&A
	\item Lots of coffee
  \end{itemize}

  \emph{May 2016 to January 2018}: \textbf{Vocalink}, Director of Data
  Science.

  \emph{January 2014 to March 2016} : \textbf{New York Times R\&D}, Data
  Scientist.

  \emph{May 2011 to December 2013} : \textbf{bitly Inc.}, Senior Data
  Scientist.

  \emph{January 2010 to April 2011} : \textbf{Columbia University},
  Postdoctoral Researcher, Department of Applied Physics and Applied
  Mathematics.

  \emph{July 2008 to December 2009} : \textbf{University of Edinburgh},
  Postdoctoral Researcher, Adaptive and Neural Computation, School of
  Informatics.

  \emph{May 2007 to June 2008} : \textbf{University of Sheffield},
  Postdoctoral Researcher, Department of Automatic Control \& Systems
  Engineering and the Department of Computer Science.

\section{Selected Projects - Mastercard}

	\href{https://www.bloomberg.com/news/articles/2023-07-05/mastercard-s-ai-tool-helps-nine-british-banks-tackle-scams}{\textbf{Consumer Fraud Risk}} Financial Crime Solutions, \textbf{Mastercard} - provides a pre-payment API to detect scams on bank to bank payments. My team and I developed, built, deployed and operated the application layer of this service. It is used by major UK banks, with TSB estimating the service will save the UK economy £100M per year.
  \begin{itemize}
  \item
    Consumer Anti-Fraud Solution of the Year - Payments Awards 2023
  \item
    Best Security or Anti-Fraud Development - The Card and Payments
    Awards 2024
  \end{itemize}
	\href{https://www.vocalink.com/news-insights/case-studies/case-study-mits/}{\textbf{Trace Financial Crime}} Financial Crime Solutions, \textbf{Mastercard} - detects money laundering over instant payments networks. This service is used by the 13 largest banks in the UK, covering well over 90\% of the UK faster payments participants. My team and I executed the techincal design, build, deployment, and subsequent operation of this service.

  \begin{itemize}
  \item
    Rising Star Award - Deloitte Market Gravity Awards 2018
  \end{itemize}
	\href{https://www.thetimes.co.uk/article/rbs-system-pushes-back-against-invoice-fraudsters-88h92l5ml}{\textbf{Corporate Fraud Insights}} Vocalink Analytics, \textbf{Mastercard} - detects fraud in the Bacs payment network in the UK. Working with RBS, this service prevented over £7MM of losses to RBS’s customers in less than two years. My team and I built the behavioural modelling, scoring mechanism and application layer to deliver this service. 

  \begin{itemize}
  \item
    Banking Security Innovation of the Year - Retail Banker
    International Awards 2018
  \item
    Analytics Project of the Year - National Technology Awards 2018
  \item
    Best Security or Anti-Fraud Development - The UK Card \& Payments
    Award 2019
  \end{itemize}
\section{Selected Projects - New York Times}
\begin{itemize}
\item
	\href{https://github.com/nytlabs/streamtools}{\textbf{streamtools}} R\&D, \textbf{New York Times}. A graphical toolkit for working with live streams of data.
\item
	\href{http://nytlabs.com/projects/editor.html}{\textbf{editor}} R\&D, \textbf{New York Times}. A prototype text editor that uses a recurrent neural network and word level embeddings to perform semi-automated tagging of sub-sentence blocks of text
\item
	\href{http://nytlabs.com/projects/lazarus.html}{\textbf{lazarus}} R\&D, \textbf{New York Times}. An application of machine vision techniques to associate a photo from the physical archive with its digital counterpart in the NYT's digital archive.
\end{itemize}

\section{Community Engagement}

\begin{itemize}
\item
	\href{https://www.bankofengland.co.uk/research/fintech/ai-public-private-forum}{\textbf{AI Public Private Forum}} 2020-2021. I attended the AI Public
  Private Forum, run by the Financial Conduct Authority and the Bank of
  England.
\item
  \textbf{NYT R\&D Data Meeting} 2013-2016. I ran a weekly, internal
  cross-departmental meeting at the NYT designed to explore the use of
  data, in all its forms, inside the NYT.
\item
  \textbf{Data Gotham} 2012-2013. I was a co-organizer of Data Gotham -
  a two day event celebrating Data Science in New York.
\item
	\href{https://www.youtube.com/watch?v=23FgdEHOt0w}{\textbf{talk: Streamtools}} 2015. A talk  I gave at code Neuro, about streamtools.
\item
	\href{http://videos.theconference.se/mike-dewar-big-data-understand-and}{\textbf{talk: Seeing From Above}} 2013. A talk I gave in Malmo, Sweden, about data science.
\item
	\href{https://www.youtube.com/watch?v=Jsg4R9z\_Z7M}{\textbf{talk: The Data Perspective}} 2015. A talk I gave at the NYC R Conference, about values 
\item
  Author of \textbf{Getting Started with D3}, Dewar M.A., O'Reilly,
  2012.
\end{itemize}

\section{Selected Academic Papers}

\begin{itemize}
\item
  \textbf{Point process modelling of the Afghan War Diary}, Andrew
  Zammit-Mangion, Michael Dewar, Visakan Kadirkamanathan, and Guido
  Sanguinetti. PNAS 2012.

  \begin{itemize}
  \item
	  \href{https://www.pnas.org/post/update/2012-cozzarelli-prize-recipients}{PNAS 2012 Cozzarelli Prize Winner} (Engineering and Applied Sciences)
  \end{itemize}
\item
  \textbf{Inference in Hidden Markov Models with Explicit State Duration
  Distributions}, M. Dewar and C. Wiggins and F. Wood. IEEE Signal
  Processing Letters, 2012.
\item
  \textbf{Parameter Estimation and Inference for Stochastic
  Reaction-Diffusion Systems: application to morphogenesis in D.
  melanogaster}, Dewar M.A., Kadirkamanathan, V., Opper, M. and
  Sanguinetti, G. BMC Systems Biology 2010, 4:21.
\item
  \textbf{Classifying \emph{Drosophila} Courtship}, Dewar M.A. Invited
  talk at Virtual Fly Brain - Behaviour Workshop. September 21-23, 2009
  at Magdalen College, Oxford.
\item
  \textbf{Estimation and Model Selection of an IDE based Spatiotemporal
  Model}, Scerri K, Dewar M.A.~and Kadirkamanathan V. IEEE Transactions
  on Signal Processing. 2009. 57(2) pp.482-492.
\item
  \textbf{Data Driven Spatiotemporal Modelling Using the
  Integro-Difference Equation}, Dewar M.A., Scerri K. and
  Kadirkamanathan V. IEEE Transactions on Signal Processing. 2009. 57(1)
  pp.83-91.
\item
  \textbf{Classification of Animal Behaviour Using Dynamic Models of
  Movement}, M.A.~Dewar, J.A. Heward, T.C. Lukins and J.D. Armstrong.
  NIPS Workshop: ``Stochastic Models of Behaviour'', 2008, Whistler.
\item
  \textbf{A Canonical Space-Time State Space Model: State and Parameter
  Estimation}, Dewar M.A.~and Kadirkamanathan V. IEEE Transactions on
  Signal Processing. 2007. 55(10) pp.4862-4870.
\end{itemize}

\end{resume}

\end{document}
