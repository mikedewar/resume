\documentclass[line, overlapped]{res}

\usepackage[dvipsnames]{xcolor}
\usepackage{hyperref}
\hypersetup{
	colorlinks=true,
	urlcolor=NavyBlue
}

\usepackage{libertine}
\renewcommand{\familydefault}{\sfdefault}

\usepackage{inconsolata}

\providecommand{\tightlist}{%
  \setlength{\itemsep}{0pt}\setlength{\parskip}{0pt}}

 \usepackage[subtle]{savetrees}

\begin{document}

\name{Dr Michael A. Dewar}

\address{
London\\
mikedewar@gmail.com
}

\begin{resume}

I build products based on algorithms, powered by data. I work with advanced techniques from data science in order to build next generation products in novel domains. As an innovation focused leader, I rely on a clear vision of the future to generate alignment in my expert teams and stakeholders.  I build trust through clear communication, transparent process, and a focus on impact.

For a full portfolio, please see \href{https://mikedewar.github.io}{mikedewar.github.io}.

\section{Education}

\emph{The University of Sheffield, UK}: PhD Thesis: \textbf{`A
  Framework for Modelling Dynamic Spatiotemporal Systems'}. Awarded June
  2007. 

\emph{The University of Sheffield, UK}: \textbf{1st Class MEng} in
  Control Systems Engineering. Awarded August 2002.

\section{Employment}

  \emph{January 2018 onwards}: \textbf{Mastercard, Cyber \&
  Intelligence}, Vice President of Data Science.
  \begin{itemize}
  \tightlist
	\item Led a cross functional team of data scientists, developers, and devops engineers, growing from 5 members to 35 at its peak.
	\item Promoted a cloud-native, prototype driven approach to design and development, tightly integrating data science and software engineering into product focused teams.
	\item Engaged in both strategy focused work and community outreach around API orchestration, Privacy Enhancing Technology, and AI.
	\item Designed synergy products as part of Mastercard's inorganic growth activities.
	\item New technologies: Homomorphic Encryption, Federated Machine Learning.
  \end{itemize}

  \emph{May 2016 to January 2018}: \textbf{Vocalink}, Director of Data Science.
  \begin{itemize}
  \tightlist
	\item Led a Data Science and Analytics team, focusing primarily on business intelligence and economic crime products. 
	\item Introduced the Data Science discipline to Vocalink, and lobbied for the modernisation of the technology stack and engineering process.
	\item Managed the team through the acquisition of Vocalink by Mastercard.
	\item New technologies: Kafka, Teradata, \texttt{xgboost}.
  \end{itemize}

  \emph{January 2014 to March 2016}: \textbf{New York Times R\&D}, Data
  Scientist.
  \begin{itemize}
  \tightlist
	  \item Provided Data Science capability to the R\&D Lab as one of five creative technologists. 
	\item Explored the impact of AI and Data Science on the future of the New York Times.
	\item High client engagement, leading tours through the lab for many different kinds of stakeholders and visitors to the New York Times to discuss the future of news.
	\item New technologies: Kubernetes, GCP, \texttt{word2vec}, \texttt{openCV}, ELK.
  \end{itemize}

  \emph{May 2011 to December 2013}: \textbf{bitly Inc.}, Senior Data Scientist.
  \begin{itemize}
  \tightlist
	\item Used bit.ly data to explore new product opportunities.
	\item Demonstrated the early ``data science hypothesis'' that data collected as a by-product could be highly valuable both internally and externally.
	\item New technologies: Docker, Golang, Python, Javascript, Redis, AWS, \texttt{d3.js}, \texttt{sckit-learn}, \texttt{pandas}, \texttt{git}.
  \end{itemize}

\newpage

  \emph{January 2010 to April 2011}: \textbf{Columbia University},
  Postdoctoral Researcher, Department of Applied Physics and Applied
  Mathematics.

  \emph{July 2008 to December 2009}: \textbf{University of Edinburgh},
  Postdoctoral Researcher, Adaptive and Neural Computation, School of
  Informatics.

  \emph{May 2007 to June 2008}: \textbf{University of Sheffield},
  Postdoctoral Researcher, Department of Automatic Control \& Systems
  Engineering and the Department of Computer Science.

\section{Selected Projects - Mastercard}

	\href{https://www.bloomberg.com/news/articles/2023-07-05/mastercard-s-ai-tool-helps-nine-british-banks-tackle-scams}{\textbf{Consumer Fraud Risk}} Financial Crime Solutions, \textbf{Mastercard} - provides a pre-payment API to detect scams on bank to bank payments. My team and I developed, built, deployed and operated the application layer of this service. It is used by major UK banks, with TSB estimating the service will save the UK economy £100MM per year.
  \begin{itemize}
  \tightlist
  \item
    Consumer Anti-Fraud Solution of the Year - Payments Awards 2023.
  \item
    Best Security or Anti-Fraud Development - The Card and Payments Awards 2024.
  \end{itemize}
	\href{https://www.vocalink.com/news-insights/case-studies/case-study-mits/}{\textbf{Trace Financial Crime}} Financial Crime Solutions, \textbf{Mastercard} - detects money laundering over instant payments networks. This service is used by the 13 largest banks in the UK, covering well over 90\% of the UK faster payments participants. My team and I executed the technical design, build, deployment, and subsequent operation of this service.

  \begin{itemize}
  \tightlist
  \item
    Rising Star Award - Deloitte Market Gravity Awards 2018.
  \end{itemize}
	\href{https://www.thetimes.co.uk/article/rbs-system-pushes-back-against-invoice-fraudsters-88h92l5ml}{\textbf{Corporate Fraud Insights}} Vocalink Analytics, \textbf{Mastercard} - detects fraud in the Bacs payment network in the UK. Working with RBS, this service prevented over £7MM of losses to RBS’s customers in less than two years. My team and I built the behavioural modelling, scoring mechanism and application layer to deliver this service. 

  \begin{itemize}
  \tightlist
  \item
    Banking Security Innovation of the Year - Retail Banker International Awards 2018.
  \item
    Analytics Project of the Year - National Technology Awards 2018.
  \item
    Best Security or Anti-Fraud Development - The UK Card \& Payments Award 2019.
  \end{itemize}
\section{Selected Projects - New York Times}
	\href{https://github.com/nytlabs/streamtools}{\textbf{streamtools}} R\&D, \textbf{New York Times}. A graphical toolkit for working with live streams of data.

	\href{http://nytlabs.com/projects/editor.html}{\textbf{editor}} R\&D, \textbf{New York Times}. A prototype text editor that uses a recurrent neural network and word level embeddings to perform semi-automated tagging of sub-sentence blocks of text.

	\href{http://nytlabs.com/projects/lazarus.html}{\textbf{lazarus}} R\&D, \textbf{New York Times}. An application of machine vision techniques to associate a photo from the physical archive with its digital counterpart in the NYT's digital archive.

\section{Community Engagement}

	\href{https://www.bankofengland.co.uk/research/fintech/ai-public-private-forum}{\textbf{AI Public Private Forum}} 2020-2021. I attended the AI Public
  Private Forum, run by the Financial Conduct Authority and the Bank of
  England.

  \textbf{NYT R\&D Data Meeting} 2013-2016. I ran a weekly, internal
  cross-departmental meeting at the NYT designed to explore the use of
  data, in all its forms, inside the NYT.

  \textbf{Data Gotham} 2012-2013. I was a co-organizer of Data Gotham -
  a two day event celebrating Data Science in New York City.

  Recorded Talks:
  \begin{itemize}
  \tightlist
	  \item \href{https://www.youtube.com/watch?v=23FgdEHOt0w}{\textbf{Streamtools}} 2015. A talk  I gave at code Neuro, about streamtools.

	  \item \href{https://www.youtube.com/watch?v=Jsg4R9z\_Z7M}{\textbf{The Data Perspective}} 2015. A talk I gave at the NYC R Conference, about values.

	  \item \href{http://videos.theconference.se/mike-dewar-big-data-understand-and}{\textbf{Seeing From Above}} 2013. A talk I gave in Malmo, Sweden, about data science.

  \end{itemize}

  Author of \textbf{Getting Started with D3}, Dewar M.A., O'Reilly,
  2012.

  Full list of academic publications available at \href{https://mikedewar.github.io}{mikedewar.github.io}.

\end{resume}

\end{document}
