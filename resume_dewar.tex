% LaTeX resume using res.sty
\documentclass{res}
\usepackage{palatino}
\setlength{\textwidth}{5.6in} % set width of text portion
\setlength{\textheight}{9.5 in}

\begin{document}

% Center the name over the entire width of resume:
 \moveleft.5\hoffset\centerline{\large\bf Michael A. Dewar}
% Draw a horizontal line the whole width of resume:
 \moveleft\hoffset\vbox{\hrule width\resumewidth height 1pt}\smallskip
% address begins here
% Again, the address lines must be centered over entire width of resume:
 \moveleft.5\hoffset\centerline{Brooklyn, NY}
 \moveleft.5\hoffset\centerline{mikedewar@gmail.com}
 \moveleft.5\hoffset\centerline{github.com/mikedewar -- twitter.com/mikedewar}

\begin{resume}
    
    \section{Profile}
    
  I build tools designed to enable the design of data-rich systems in novel situations. My focus, and the focus that I encourage in others, is to use algorithms that can be clearly understood and interpreted in the organisations that deploy them. 
    
    \section{Education}
 
\begin{itemize}
    \item \emph{The University of Sheffield, UK}: PhD Thesis: {\bf `A Framework for Modelling Dynamic Spatiotemporal Systems'}. Awarded June 2007.
    \item \emph{The University of Sheffield, UK}: {\bf 1st Class MEng} in Control Systems Engineering. Awarded August 2002.
\end{itemize}

\section{Employment} 

\begin{itemize}
    \item \emph{January 2014 onwards} : \textbf{New York Times R\&D}, Data Scientist.
    \item \emph{May 2011 to December 2013} : \textbf{bitly Inc.}, Senior Data Scientist.
    \item \emph{January 2010 to April 2011} : \textbf{Columbia University},
	Postdoctoral Researcher, Department of Applied Physics and Applied Mathematics.
	\item \emph{July 2008 to December 2009} : \textbf{University of Edinburgh}, Postdoctoral Researcher, Adaptive and Neural Computation, School of Informatics.
	\item \emph{May 2007 to June 2008} : \textbf{University of Sheffield}, Postdoctoral Researcher, Department of Automatic Control \& Systems Engineering and the Department of Computer Science.
\end{itemize}

\section{Selected Projects and Outcomes}
\vspace{1em}
\textbf{Realtime Monitoring of Web-based Systems} (New York 2011 onwards) My work at bitly and The New York Times involves finding and exposing value in the data assets collected through the behaviour of large online media audiences. This work is highly varied, involving one-off analysis projects, product prototyping, infrastructure development, and tool building. My work has focused mainly on online stream processing applications, the dominant online data structure.
\begin{itemize}
    \item \textbf{Streamtools: https://github.com/nytlabs/streamtools} NYT R\&D. 2014-2015. Streamtools is an open source, graphical toolkit for dealing with live streams of data. Its aim was to allow analysts and designers build algorithms that work directly on a stream of data, rather than performing offline processing at a later date.
    \item \textbf{Bitly Science: http://bitlyscience.github.com} Bitly Science Team. 2011-2012. This website showcases a number of blog posts and magazine articles created using analysis from the bitly science team. My role has been to do the analysis for a number of posts, as well as work with teams in the mainstream media, most notably The Guardian and Scientific American.
\end{itemize}

\textbf{Prototyping} (New York 2013-) A lot of my work at the New York Times involves contributing to the speculative prototyping work the lab performs. Specifically I have contributed to
\begin{itemize}
\item \textbf{editor: http://nytlabs.com/projects/editor.html} NYT R\&D. 2014-2015. A prototype text editor that uses a recurrent neural network to perform semi-automated tagging of sub-sentence blocks of text.
\item \textbf{lazarus: http://nytlabs.com/projects/lazarus.html} NYT R\&D. 2013-2014. A system that uses some basic machine vision techniques to associate a photo from the physical archive with its digital counterpart in the NYT's digital archive.
\item \textbf{colony: https://github.com/nytlabs/colony} NYT R\&D. 2014. A microservice messaging framework for NSQ (http://nsq.io/), used to explore what kind of systems are afforded by distributed computation and deployment.
\end{itemize}
    
    \textbf{Spatiotemporal Modelling} (Sheffield, Edinburgh, New York 2003-2011) - This project began with my PhD Thesis, which was focused on learning linear dynamic models from spatiotemporal data. The main focus of this work is learning models which are interpretable in terms of the underlying system. My work has been followed up on in two main projects: neural field modelling and an analysis of the Afghanistan Wikileaks data set.

    \begin{itemize}
        \item {\bf Point process modelling of the Afghan War Diary}, Andrew Zammit-Mangion, Michael Dewar, Visakan Kadirkamanathan, and Guido Sanguinetti. PNAS 2012.
        \item {\bf A Data-Driven Framework for Neural Field Modelling}, D. R. Freestone, P. Aram, M. Dewar, K. Scerri, D. B. Grayden, and V. Kadirkamanathan. Neuroimage, 2011.
        \item {\bf Parameter Estimation and Inference for Stochastic Reaction-Diffusion Systems: application to morphogenesis in D. melanogaster}, Dewar M.A., Kadirkamanathan, V., Opper, M. and Sanguinetti, G. BMC Systems Biology 2010, 4:21.
        \item {\bf Modelling Spatiotemporal Systems using the Integrodifference Equation}, Dewar M.A., Invited talk at Information: Signals, Images, Systems seminar series, University of Southampton, 2009.
        \item {\bf Estimation and Model Selection of an IDE based Spatiotemporal Model}, Scerri K, Dewar M.A. and Kadirkamanathan V. IEEE Transactions on Signal Processing. 2009. 57(2) pp.482-492.
        \item {\bf Data Driven Spatiotemporal Modelling Using the Integro-Difference Equation}, Dewar M.A., Scerri K. and Kadirkamanathan V. IEEE Transactions on Signal Processing. 2009. 57(1) pp.83-91. 
        \item {\bf A Canonical Space-Time State Space Model: State and Parameter Estimation}, Dewar M.A. and Kadirkamanathan V. IEEE Transactions on Signal Processing. 2007. 55(10) pp.4862-4870.
    %\item {\bf Modelling of the oesophagus using spatiotemporal observations}, Dewar M.A., Fabri S.G., Kadirkamanathan V., 2nd International Conference on Advances in Medical Signal Processing, 2004, Malta.
    \end{itemize}

\textbf{Modelling Behaviour} (Edinburgh and New York 2008 - 2014) - This work, the Edinburgh portion of which resulted in a successful startup called Actual Analytics, seeks to automate routine animal behavioural analysis from collected video data. I continued the theoretical aspects of this work at Columbia University, and scaled up to large online audiences at The New York Times.
    \begin{itemize}
        \item {\bf Inference in Hidden {M}arkov Models with Explicit State Duration Distributions}, M. Dewar and C. Wiggins and F. Wood. IEEE Signal Processing Letters, 2012.
        \item {\bf Classification of Animal Behaviour Using Dynamic Models of Movement},     M.A. Dewar, J.A. Heward, T.C. Lukins and J.D. Armstrong. NIPS Workshop: ``Stochastic Models of Behaviour'', 2008, Whistler. 
        \item {\bf iBehave: Towards Sequencing Behaviour}, Heward J.A., Lukins T., Dewar M.A., Armstrong J.D., Measuring Behaviour, 2008, Maastricht.
        \item {\bf Classifying Active Investigation}, Lukins T., Dewar M.A., Crook P., Hawcock T., Armstrong J.D., Measuring Behaviour, 2008, Maastricht.
        %\item {\bf The Application of Hidden Markov Models to Behavioural Data}, Dewar M.A. Invited talk at Rules, Relations and Robotics: Making the most of behavioural data, September 2009. Royal Veterinary College.
	\item {\bf Classifying \emph{Drosophila} Courtship}, Dewar M.A. Invited talk at Virtual Fly Brain - Behaviour Workshop. September 21-23, 2009 at Magdalen College, Oxford.
    \end{itemize}

\section{Community Engagement}

A large part of my work involves engaging with and sometimes building the communities around the disciplines I work within. 
\begin{itemize}
  \item {\bf NYT R\&D Data Meeting} 2013-. I run a weekly, internal cross-departmental meeting at the NYT designed to explore the use of data, in all its forms, inside the NYT. 
  \item {\bf Data Gotham} 2012-2013. I was a co-organizer of Data Gotham - a two day event celebrating Data Science in New York. http://www.datagotham.com/
  \item {\bf talk: Seeing From Above} 2013. A talk I gave in Malmo, Sweden, about data science http://videos.theconference.se/mike-dewar-big-data-understand-and 
    \item {\bf talk: The Data Perspective} 2015. A talk I gave at the NYC R Conference, about values https://www.youtube.com/watch?v=Jsg4R9z_Z7M 

  \item {\bf Meetups} 2011-. I talk semi-regularly at Meetups, including the Machine Learning, Open Statistical Programming, and Data Community DC meetings.
  \item {\bf PASCAL2 Workshop on Spatiotemporal Modelling} 2009. I ran a small workshop on spatiotemporal modelling at Edinburgh University. http://www.pascal-network.org/?q=node/153.
  \item Author of {\bf Getting Started with D3}, Dewar M.A., O'Reilly, 2012.  
\end{itemize}

\end{resume}
\end{document}
